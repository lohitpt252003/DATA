% ANNAFORCES - Solution Template for Problem P8 (Three Segments)
% Save as: ANNAFORCES_P8_Solution.tex
\documentclass[11pt,a4paper]{article}
\usepackage[utf8]{inputenc}
\usepackage[T1]{fontenc}
\usepackage{geometry}
\usepackage{amsmath,amssymb}
\usepackage{fancyhdr}
\usepackage{enumitem}
\usepackage{graphicx}
\usepackage{hyperref}
\usepackage{listings}
\usepackage{xcolor}
\geometry{margin=1in}
\pagestyle{fancy}
\fancyhf{}
\lhead{\textbf{ANNAFORCES}}
\rhead{Solution — P8}
\cfoot{\thepage}

% Listings setup for code blocks
\lstset{
  basicstyle=\ttfamily\small,
  breaklines=true,
  frame=single,
  numbers=left,
  numberstyle=\tiny,
  showstringspaces=false,
  tabsize=2
}

\begin{document}

\begin{center}
  {\LARGE \bf Solution: Three Segments (P8)}\\[6pt]
  {\large \it ANNAFORCES}\
  \vspace{6pt}
\end{center}

\section*{Problem Description}
You are given an array of integers `arr` of length `n`.
You need to partition the array into exactly **three contiguous segments**.

Let:

* $s1 = \left(\text{sum of elements in the first segment}\right) \bmod 3$
* $s2 = \left(\text{sum of elements in the second segment}\right) \bmod 3$
* $s3 = \left(\text{sum of elements in the third segment}\right) \bmod 3$

Your task is to print:

* `YES` if either

  1. $s1 = s2 = s3$, or
  2. $s1, s2, s3$ are all **pairwise different** (no two are equal).
* Otherwise, print `NO`.

\section*{Simple Answer}
The solution checks if the total sum of the array is divisible by 3, which is a necessary and sufficient condition for a valid partition to exist.

\section*{Detailed Explanation}
\subsection*{Approach}
The solution relies on a key mathematical observation about the sums of the segments modulo 3.

Let the three contiguous segments have sums `sum1`, `sum2`, and `sum3`. Let the modulo values be `s1 = sum1 % 3`, `s2 = sum2 % 3`, and `s3 = sum3 % 3`. The total sum of the array is `S = sum1 + sum2 + sum3`.

The sum of the modulo values, `(s1 + s2 + s3) % 3`, must be equal to the total sum modulo 3, `S % 3`.

Let's analyze the two valid conditions:
\begin{enumerate}
  	item `s1 = s2 = s3 = s`: In this case, `(s1 + s2 + s3) % 3 = (3 * s) % 3 = 0`.
  	item `{s1, s2, s3}` is a permutation of `{0, 1, 2}`: In this case, `(s1 + s2 + s3) % 3 = (0 + 1 + 2) % 3 = 3 % 3 = 0`.
\end{enumerate}

In both valid cases, the sum of the modulo values is 0. This implies that a necessary condition for a valid partition to exist is that the total sum of the array must be divisible by 3.
`S % 3 = (s1 + s2 + s3) % 3 = 0`.

It turns out that this condition is not only necessary but also sufficient for an array of length `n >= 3`. The proof relies on the fact that if the total sum is divisible by 3, we have enough flexibility with two cut points to always find a valid partition. We can either find a prefix that sums to 0 mod 3, creating a `(0, s, -s)` partition which is always valid, or we can find two prefixes that create one of the other valid partitions.

Therefore, the solution simplifies to checking if the total sum of the array is divisible by 3.

\section*{Complexity}
\begin{itemize}[leftmargin=*] 
  		item \textbf{Time Complexity:} $O(n)$ — to calculate the sum of the array.
  		item \textbf{Space Complexity:} $O(1)$ — for the Python solution, or O(n) to store the array if not processed in a stream.
\end{itemize}

\section*{Language-Specific Implementations}
Below are sample implementations in Python, C++, and C.

\subsection*{Python (solution.py)}
\begin{lstlisting}[language=Python]
import sys

def solve():
    try:
        # Read the number of elements
        n_str = sys.stdin.readline()
        if not n_str: return
        n = int(n_str)
        
        # Read the array elements
        arr = list(map(int, sys.stdin.readline().split()))

        # Calculate the total sum of the array
        total_sum = sum(arr)

        if total_sum % 3 == 0:
            print("YES")
        else:
            print("NO")

    except (IOError, ValueError):
        return

solve()
\end{lstlisting}

\subsection*{C++ (solution.cpp)}
\begin{lstlisting}[language=C++]
#include <iostream>
#include <vector>
#include <numeric>

int main() {
    std::ios_base::sync_with_stdio(false);
    std::cin.tie(NULL);
    int n;
    while (std::cin >> n) {
        std::vector<long long> arr(n);
        long long total_sum = 0;
        for (int i = 0; i < n; ++i) {
            std::cin >> arr[i];
            total_sum += arr[i];
        }

        if (total_sum % 3 == 0) {
            std::cout << "YES\n";
        } else {
            std::cout << "NO\n";
        }
    }
    return 0;
}
\end{lstlisting}

\subsection*{C (solution.c)}
\begin{lstlisting}[language=C]
#include <stdio.h>

int main() {
    int n;
    while (scanf("%d", &n) != EOF) {
        long long total_sum = 0;
        for (int i = 0; i < n; ++i) {
            long long val;
            scanf("%lld", &val);
            total_sum += val;
        }

        if (total_sum % 3 == 0) {
            printf("YES\n");
        } else {
            printf("NO\n");
        }
    }
    return 0;
}
\end{lstlisting}

\end{document}