% ANNAFORCES - Solution Template for Problem P1 (Addition)
% Save as: ANNAFORCES_P1_Solution.tex
\documentclass[11pt,a4paper]{article}
\usepackage[utf8]{inputenc}
\usepackage[T1]{fontenc}
\usepackage{geometry}
\usepackage{amsmath,amssymb}
\usepackage{fancyhdr}
\usepackage{enumitem}
\usepackage{graphicx}
\usepackage{hyperref}
\usepackage{listings}
\usepackage{xcolor}
\geometry{margin=1in}
\pagestyle{fancy}
\fancyhf{}
\lhead{\textbf{ANNAFORCES}}
\rhead{Solution — P1}
\cfoot{\thepage}

% Listings setup for code blocks
\lstset{
  basicstyle=\ttfamily\small,
  breaklines=true,
  frame=single,
  numbers=left,
  numberstyle=\tiny,
  showstringspaces=false,
  tabsize=2
}

\begin{document}

\begin{center}
  {\LARGE \bf Solution: Add (P1) — Sum of Two Integers}\\[6pt]
  {\large \it ANNAFORCES}\
  \vspace{6pt}
\end{center}

\section*{Problem Description}
Read two integers $a$ and $b$, and print their sum $a+b$.

\section*{Simple Answer}
Read the two numbers and output their sum. This is a direct implementation of integer addition.

\section*{Detailed Explanation}
\subsection*{Approach}
The operation requires three logical steps:
\begin{enumerate}[leftmargin=*]
  \item \textbf{Input Acquisition:} Read integers $a$ and $b$ from standard input.
  \item \textbf{Computation:} Compute $s = a + b$ using the language's integer addition operator.
  \item \textbf{Output Presentation:} Print $s$ to standard output followed by a newline.
\end{enumerate}

This is an $O(1)$ time and $O(1)$ additional space algorithm (only constant extra memory is used).

\subsection*{Constraints and Data Types}
Constraints: $|a|,|b| < 10^{17}$. The sum may be as large as $2\times(10^{17}-1)$ in magnitude, which fits within the signed 64-bit integer range (\texttt{long long} in C/C++). Python's native integers are arbitrary-precision and are safe as well.

\section*{Complexity}
\begin{itemize}[leftmargin=*]
  \item \textbf{Time Complexity:} $O(1)$ — only a fixed number of operations are performed.
  \item \textbf{Space Complexity:} $O(1)$ — only a few variables are used.
\end{itemize}

\section*{Language-Specific Implementations}
Below are sample implementations in Python, C++, and C. Each program reads two integers from standard input and prints their sum.

\subsection*{Python (solution.py)}
\begin{lstlisting}[language=Python]
# solution.py
# Reads two integers and prints their sum

a, b = map(int, input().split())
print(a + b)
\end{lstlisting}

\subsection*{C++ (solution.cpp)}
\begin{lstlisting}[language=C++]
// solution.cpp
// Uses long long to safely handle values up to 10^17
#include <iostream>

int main() {
    long long a, b;
    std::cin >> a >> b;
    std::cout << (a + b) << std::endl;
    return 0;
}
\end{lstlisting}

\subsection*{C (solution.c)}
\begin{lstlisting}[language=C]
/* solution.c */
#include <stdio.h>

int main() {
    long long a, b;
    if (scanf("%lld %lld", &a, &b) == 2) {
        printf("%lld\n", a + b);
    }
    return 0;
}
\end{lstlisting}

\section*{Notes and Verification}
The sample cases are simple arithmetic checks:
\begin{itemize}[leftmargin=*]
  \item Input: \texttt{1 1} -> Output: \texttt{2}
  \item Input: \texttt{2 2} -> Output: \texttt{4}
\end{itemize}
These follow directly from the definition of addition.

\end{document}
