% ANNAFORCES - Solution Template for Problem P6 (Odd or Even)
% Save as: ANNAFORCES_P6_Solution.tex
\documentclass[11pt,a4paper]{article}
\usepackage[utf8]{inputenc}
\usepackage[T1]{fontenc}
\usepackage{geometry}
\usepackage{amsmath,amssymb}
\usepackage{fancyhdr}
\usepackage{enumitem}
\usepackage{graphicx}
\usepackage{hyperref}
\usepackage{listings}
\usepackage{xcolor}
\geometry{margin=1in}
\pagestyle{fancy}
\fancyhf{}
\lhead{\textbf{ANNAFORCES}}
\rhead{Solution — P6}
\cfoot{\thepage}

% Listings setup for code blocks
\lstset{
  basicstyle=\ttfamily\small,
  breaklines=true,
  frame=single,
  numbers=left,
  numberstyle=\tiny,
  showstringspaces=false,
  tabsize=2
}

\begin{document}

\begin{center}
  {\LARGE \bf Solution: Odd or Even (P6)}\\[6pt]
  {\large \it ANNAFORCES}\
  \vspace{6pt}
\end{center}

\section*{Problem Description}
The task is to determine whether a given number is odd or even for multiple test cases.
If the number is divisible by 2, it is EVEN; otherwise, it is ODD.

\section*{Simple Answer}
Check the remainder when dividing the number by 2:
\begin{itemize}
  \item If remainder is 0 → EVEN
  \item Otherwise → ODD
\end{itemize}

\section*{Detailed Explanation}
\subsection*{Approach 1: Modulo Operation}
\begin{enumerate}
  \item \textbf{Input Acquisition:}
    \begin{itemize}
      \item First read the integer $t$, the number of test cases.
      \item For each test case, read an integer $n$.
    \end{itemize}
  \item \textbf{Computation:}
    \begin{itemize}
      \item Use the modulo operation (`n % 2`).
      \item If the remainder is 0, print "EVEN"; otherwise, print "ODD".
    \end{itemize}
  \item \textbf{Output Presentation:}
    \begin{itemize}
      \item For each test case, print the result on a new line.
    \end{itemize}
\end{enumerate}

\subsection*{Approach 2: Bitwise Operation}
We can also determine odd/even using the **last bit of the binary representation**:
\begin{itemize}
  \item In binary, **even numbers** always end with `0`
  \item **Odd numbers** always end with `1`
\end{itemize}
So:
\begin{lstlisting}[language=C++]
if (n & 1ll) cout << "ODD";
else cout << "EVEN";
\end{lstlisting}
Here `n & 1ll` checks the least significant bit (LSB) of `n`. If it is 1 → odd, else → even.
This is often **faster** than modulo.

\subsection*{Constraints and Data Types}
$1 \leq t \leq 10^5$, $0 \leq n \leq 10^{18}$.
Since $n$ can be very large, we must use 64-bit integers (`long long` in C/C++). Python naturally supports big integers, so no special care is required.

\section*{Language-Specific Implementations}
Below are sample implementations in Python, C++, and C.

\subsection*{Python (solution.py)}
\begin{lstlisting}[language=Python]
t = int(input())  # number of testcases
for _ in range(t):
    n = int(input())  # read number
    if n % 2 == 0:
        print("EVEN")  # divisible by 2
    else:
        print("ODD")   # not divisible by 2
\end{lstlisting}

\subsection*{C++ (solution.cpp)}
\begin{lstlisting}[language=C++]
#include <iostream>
using namespace std;

int main() {
    int t;
    cin >> t;  // read number of testcases
    while (t--) {
        long long n;
        cin >> n;  // read number
        if (n & 1ll) cout << "ODD" << endl;   // bitwise check
        else cout << "EVEN" << endl;
    }
    return 0;
}
\end{lstlisting}

\subsection*{C (solution.c)}
\begin{lstlisting}[language=C]
#include <stdio.h>

int main() {
    int t;
    scanf("%d", &t);  // number of testcases
    while (t--) {
        long long n;
        scanf("%lld", &n);  // read number
        if (n & 1ll) printf("ODD\n");  // bitwise check
        else printf("EVEN\n");
    }
    return 0;
}
\end{lstlisting}

\end{document}