\documentclass[12pt]{article}
\usepackage{amsmath}
\usepackage{amssymb}
\usepackage{listings}
\usepackage{xcolor}

% Code formatting style
\lstset{
  basicstyle=\ttfamily\small,
  backgroundcolor=\color{gray!10},
  frame=single,
  breaklines=true,
  showstringspaces=false
}

\begin{document}

\section*{Solution: Odd or Even}

\textbf{Approach:} \\
We can solve this problem using either the modulo operator or bitwise operations.

\begin{enumerate}
    \item \textbf{Modulo Method:}  
    Check $n \bmod 2$:
    \begin{itemize}
        \item If $n \bmod 2 = 0$, then $n$ is even.
        \item Otherwise, $n$ is odd.
    \end{itemize}

    \item \textbf{Bitwise Method:}  
    Observe that the least significant bit (LSB) of $n$ determines odd/even:
    \begin{itemize}
        \item If $(n \,\&\, 1ll) = 0$, then $n$ is even.
        \item If $(n \,\&\, 1ll) = 1$, then $n$ is odd.
    \end{itemize}
\end{enumerate}

\textbf{Python Implementation:}
\begin{lstlisting}[language=Python]
t = int(input())
for _ in range(t):
    n = int(input())
    if n % 2 == 0:
        print("EVEN")
    else:
        print("ODD")
\end{lstlisting}

\textbf{C++ Implementation (bitwise):}
\begin{lstlisting}[language=C++]
#include <iostream>
using namespace std;

int main() {
    int t;
    cin >> t;
    while (t--) {
        long long n;
        cin >> n;
        if (n & 1ll) cout << "ODD\n";
        else cout << "EVEN\n";
    }
    return 0;
}
\end{lstlisting}

\textbf{C Implementation (bitwise):}
\begin{lstlisting}[language=C]
#include <stdio.h>

int main() {
    int t;
    scanf("%d", &t);
    while (t--) {
        long long n;
        scanf("%lld", &n);
        if (n & 1ll) printf("ODD\n");
        else printf("EVEN\n");
    }
    return 0;
}
\end{lstlisting}

\end{document}
