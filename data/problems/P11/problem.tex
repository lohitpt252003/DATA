% ================================================================
% == ANNAFORCES - Reusable Problem Statement Template (v4) ==
% ================================================================
% Changelog:
% v4: - Used a `description` list for Input/Output/Constraints
%       to fix alignment and improve structure.
% v3: - Used a `tabular` environment for metadata alignment.
% v2: - Added `itemize` for multiple constraints.
%     - Renamed "verification" to "Explanation".
% ================================================================

\documentclass[11pt,a4paper]{article}

% --- Standard Packages ---
\usepackage[utf8]{inputenc}
\usepackage[T1]{fontenc}
\usepackage{amsmath,amssymb}
\usepackage{geometry}
\usepackage{fancyhdr}
\usepackage{hyperref}
\usepackage[shortlabels]{enumitem} % Added shortlabels option for description list
\usepackage{graphicx}
\usepackage{tcolorbox}
\usepackage{verbatim}

% --- Page Layout ---
\geometry{margin=1in}
\pagestyle{fancy}
\fancyhf{}
\lhead{\textbf{ANNAFORCES}}
\rhead{Problem Statement}
\cfoot{\thepage}

% --- Custom Commands for Problem Structure ---
% Note: Using description list for Input/Output/Constraints now
\newcommand{\ProblemTitle}[1]{\begin{center}\LARGE\bfseries #1\end{center}\vspace{6pt}}
\newcommand{\ProblemID}[1]{\noindent\textbf{Problem ID:} #1\\}
\newcommand{\Statement}[1]{\vspace{6pt}\begin{tcolorbox}[colback=white,colframe=black!60,boxrule=0.5pt]\noindent #1\end{tcolorbox}\vspace{8pt}}

% ------------------------------
\begin{document}

% ================================================================
% == PROBLEM DETAILS SECTION ==
% == Fill in all the information below. ==
% ================================================================

% --- Problem Title (large, centered) ---
\ProblemTitle{[Problem Title Here]}

% --- Problem Metadata (Aligned in a table) ---
% Edit the right-hand column values below as needed.
\begin{tabular}{@{} l @{\hspace{1em}} p{0.72\linewidth}@{}}
    \textbf{Problem ID:} & \textbf{[Problem ID]} \\
    \textbf{Contest ID:} & \textbf{[Contest ID (If available)]} \\
    \textbf{Difficulty:} & [Difficulty Level: e.g., Easy, Medium, Hard] \\
    \textbf{Tags:}       & [Comma-separated tags: e.g., math, implementation, dp] \\
    \textbf{Authors:}    & [Comma-separated names: e.g., Baz, Boss] \\
\end{tabular}
\vspace{8pt}

% --- Main Problem Body ---
\Statement{
[Problem statement text goes here. You can use LaTeX math mode for variables, like $N$ or $x_i$. Describe the task the user needs to solve.]
}

% --- Constraints (separate section, outside the description list) ---
% --- Input, Output, and Constraints Section (Aligned with description and tabular) ---
\begin{description}[style=unboxed, leftmargin=0pt, font=\normalfont\textbf]
    \item[Input Format:] [Describe the input format line by line.]
    \item[Output Format:] [Describe the expected output format.]
    \item[Constraints:]
        \begin{tabular}{@{} l @{\hspace{1em}} l}
            $\bullet$ [Description of constraint 1] & [e.g., $1 \le N \le 10^5$] \\
            $\bullet$ [Description of constraint 2] & [e.g., $0 \le A_i \le 10^9$] \\
            $\bullet$ [Add more as needed] & [Constraint details] \\
        \end{tabular}
\end{description}
\vspace{8pt}

% --- Sample Cases ---
\noindent\textbf{Sample Input 1:}
\begin{verbatim}
[Paste Sample 1 Input Here]
\end{verbatim}

\noindent\textbf{Sample Output 1:}
\begin{verbatim}
[Paste Sample 1 Output Here]
\end{verbatim}

\noindent\textbf{Sample Input 2:}
\begin{verbatim}
[Paste Sample 2 Input Here]
\end{verbatim}

\noindent\textbf{Sample Output 2:}
\begin{verbatim}
[Paste Sample 2 Output Here]
\end{verbatim}

% --- Optional Sections ---
\noindent\textbf{Notes:}
[Add any hints, clarifications, or extra information here.]

\vspace{8pt}
% --- Explanation Section for Sample Cases ---
\noindent\textbf{Explanation:}
\begin{itemize}[leftmargin=*, topsep=2pt, partopsep=0pt]
  \item \textbf{Sample 1:} [Explain why the input for Sample 1 produces the given output.]
  \item \textbf{Sample 2:} [Explain why the input for Sample 2 produces the given output.]
\end{itemize}

\end{document}
