% LaTeX Problem Statement Template - Addition Problem (ANNAFORCES)
% Save as: annaforces_add_problem.tex
\documentclass[11pt,a4paper]{article}
\usepackage[utf8]{inputenc}
\usepackage[T1]{fontenc}
\usepackage{amsmath,amssymb}
\usepackage{geometry}
\usepackage{fancyhdr}
\usepackage{hyperref}
\usepackage{enumitem}
\usepackage{graphicx}
\usepackage{tcolorbox}
\usepackage{verbatim}
\geometry{margin=1in}
\pagestyle{fancy}
\fancyhf{}
\lhead{\textbf{ANNAFORCES}}
\rhead{Problem Statement}
\cfoot{\thepage}

% ----- Metadata macros -----
\newcommand{\ProblemTitle}[1]{\begin{center}\LARGE\bfseries #1\end{center}\vspace{6pt}}
\newcommand{\Difficulty}[1]{\noindent\textbf{Difficulty:} #1\\}
\newcommand{\Tags}[1]{\noindent\textbf{Tags:} #1\\}
\newcommand{\Authors}[1]{\noindent\textbf{Authors:} #1\\}
\newcommand{\Statement}[1]{\vspace{6pt}\begin{tcolorbox}[colback=white,colframe=black!60,boxrule=0.5pt]\noindent #1\end{tcolorbox}\vspace{8pt}}
\newcommand{\InputFormat}[1]{\noindent\textbf{Input Format:} #1\\}
\newcommand{\OutputFormat}[1]{\noindent\textbf{Output Format:} #1\\}
\newcommand{\Constraints}[1]{\noindent\textbf{Constraints:} #1\\}

% ------------------------------
\begin{document}

% --- Fill metadata here ---
\ProblemTitle{Add (P1) — Sum of Two Integers}
\Difficulty{Easy}
\Tags{easy, input, output}
\Authors{boss, baz}

\Statement{
Given two integers $a$ and $b$, print their sum $a + b$.
}

\InputFormat{The first line contains two integers $a$ and $b$ separated by a space.}
\OutputFormat{Print a single integer: the sum $a + b$.}
\Constraints{$|a|,|b| < 10^{17}$.}

\noindent\textbf{Sample Input 1:}
\begin{verbatim}
1 1
\end{verbatim}
\noindent\textbf{Sample Output 1:}
\begin{verbatim}
2
\end{verbatim}

\noindent\textbf{Sample Input 2:}
\begin{verbatim}
2 2
\end{verbatim}
\noindent\textbf{Sample Output 2:}
\begin{verbatim}
4
\end{verbatim}

\noindent\textbf{Notes:}  The operation uses standard integer addition. The sample cases show that $1+1=2$ and $2+2=4$.

\vspace{8pt}
\noindent\textbf{Small verification (informal):}
\begin{itemize}[leftmargin=*]
  \item $1+1=2$: By the usual construction of natural numbers (Peano-style), $1$ is the successor of $0$, denoted $S(0)$. Then $1+1=S(0)+S(0)=S(S(0))$, which we denote as $2$.
  \item $2+2=4$: Similarly, $2=S(S(0))$. Then $2+2=S(S(0))+S(S(0))=S(S(S(S(0))))$, denoted $4$.
\end{itemize}

\end{document}
